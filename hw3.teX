\documentclass[10pt, AMS Euler]{article}
\textheight=9.25in \textwidth=7in \topmargin=-.75in
\oddsidemargin=-0.25in
\evensidemargin=-0.25in
\usepackage{url}  % The bib file uses this
\usepackage{graphicx} %to import pictures
\usepackage{amsmath, amssymb, multicol}
\usepackage{theorem, concrete}


\setlength{\intextsep}{5mm} \setlength{\textfloatsep}{5mm}
\setlength{\floatsep}{5mm}


{\theorembodyfont{\rmfamily}
	\newtheorem{definition}{Definition}[section]}
{\theorembodyfont{\rmfamily} \newtheorem{example}{Example}[section]}
{\theorembodyfont{\rmfamily} \newtheorem{lemma}{Lemma}[section]}
{\theorembodyfont{\rmfamily} \newtheorem{theorem}{Theorem}[section]}
{\theorembodyfont{\rmfamily} \newenvironment{proof}{\par{\it
			Proof:}}{\nopagebreak[4]\rule{2mm}{2mm}}}



%%%%  SHORTCUT COMMANDS  %%%%
\newcommand{\ds}{\displaystyle}
\newcommand{\Z}{\mathbb{Z}}
\newcommand{\arc}{\rightarrow}
\newcommand{\R}{\mathbb{R}}
\newcommand{\N}{\mathbb{N}}
\newcommand{\Q}{\mathbb{Q}}

%%%%  footnote style %%%%

\renewcommand{\thefootnote}{\fnsymbol{footnote}}

\pagestyle{empty}
\begin{document}
		
\noindent{\bf Homework \#3:}   
Please respond to as many of the prompts below as you can, and please attempt to precisely and fluently kick their trash.\\
\noindent{\bf Rylei Mindrum | A02352206} \\
\noindent\rule{\textwidth}{0.4pt}
    \begin{enumerate}
	\item Please prove that the sets $E = \{\mbox{even integers}\}$, $\N$, and $\Z$ are infinite sets and that they have the same cardinality.
        \item[\bf{Claim}]: The sets $E = \{\mbox{even integers}\}$, $\N$, and $\Z$ are infinite sets and all have the same cardinality.
            \begin{proof}
                \item[\bf{Proof}:]
                    First I will start by defining pieces of the statement above. 
                    \begin{itemize}
                        \item Cardinality - (set theory) Of a set, the number of elements it contains. (from Wiktionary)
                        \item Bijection - function that is a one-to-one mapping of sets. Has to satisfy Injection and Surjection. 
                        \item An Infinite Set can be defined as "not finite." Meaning that its elements can't be counted in a set amount of time or to a specific equivalency.
                        \item The Set of Even Integers, $E = \{\mbox{...,-4,-2,0,2,4,...}\}$, can be written as for every integer $\emph{k} \in \Z$ an even integer exists for \emph{2k}. And since the set of integers has been defined as infinite, every integer can be paired with it own unique even integer, so the set of $\emph{E}$ must also be infinite. 
                        \item Natural Numbers $\N$, $N = {1,2,3,..}$ is infinite by definition, and wikipedia, They are countably infinite because we can list them in a sequence without any end. 
                        \item Integers $\Z$, $Z = {...,-2,-1,0,1,2,...}$ is also infinite because for any integer \emph{k}, there exists an integer for \emph{k}+1. This shows that there is no largest element. Similar to natural number the integers are countably infinite.
                    \end{itemize}  
                    Now that the pieces of the equation are explained in their own definitions, I will prove that the cardinality of $\emph{E}, \N, and \Z$ is all the same. I will do this by using bijection. 
                    \begin{itemize}
                        \item Bejection between $\N$ and $\emph{E}$:
                        \begin{itemize}
                            \item Consider f : $\N -> \emph{E}$ defined by $f(n) = 2n$. This function will map each natural number (defined as n) to and even number (2n).
                            \\ Based on the definition of bijection the function has to also satisfy bijectioin and surjection which I will complete below:
                            \item Injection: If $f(n_{1} = f(n_{2}$, then $2n_{1} = 2n_{2}$ which implies that $n_{1} = n_{2}$. Making f injective. 
                            \item Subjection: For every integer m in the set of \emph{E} there is a natural number $n = \emph{m}/2$ such that $f(\emph{n}) = \emph{m}$. making f surjective. 
                        \end{itemize}   
                    \end{itemize}
                    since $\emph{f}$ if injective and subjective, it can be defined as a bijection. So, $\N$ and $\emph{E}$ have the same cardinality. \\
                    Now I will show that there is also bijection between $\N$ and $\Z$. 
                    \begin{itemize}
                        \item bijection between $\N$ and $\Z$:
                        \begin{itemize}
                            \item For the function $g : \N -> \Z$ we define:
                            \begin{center}
                                $g(\emph{n}) = \{{\emph{n}/2 \atop -(\emph{n}-1)/2}$
                            \end{center}
                            \item with the top being if \emph{n} is even and the bottom if \emph{n} is odd. 
                            \item The purpose of the function is to map each natural number (\emph{n}) to integer (\z) which should alternate between being positive and negative. 
                            \item Like before we have to show injective and surjective. starting with surjective: So, if $g(n_{1}) = g(n_{2})$, the cases can be distinguished based on whether $n_{1} and n_{2}$ are even or odd. In any case equality will imply that $n_{1} = n_{2}$. Meaning that g is injective. 
                            \item Now for Surjective: That is for any integer z in the set of $\Z$ we can find an n in the set of $\N$ so that g(n) = z. For example cause thats a confusing sentence:
                            \begin{itemize}
                                \item if $z >= 0$, then $n = 2z$.  Or if $z < 0$ then $n = -2z-1$. 
                            \end{itemize}
                            \item this showcases g as surjective so it is bijective. meaning that $\N$ and $\Z$ has the same cardinality. 
                        \end{itemize}   
                        \item[\bf{Conclusion}:]
                        Since $\N$ has the same cardinality as both $\emph{E}$ and $\Z$, proven by the bijections, I know that $\emph{E}$, $\Z$, and $\N$ all have the same cardinality. Shows as $|\emph{E}| = |\Z| = |\N|$ Meaning that all of the infinite sets have the same cardinality.
                    \end{itemize}
            \end{proof} \\
\noindent\rule{\textwidth}{0.4pt}
	\item Please prove that the set $(0,1) = \{x \in \R: 0 < x < 1\}$ is infinite, but does not have the same cardinality as $\N$ (and hence not the same as $E$ and $\Z$).
        \item[\bf{Claim}]: The set $(0,1) = {\emph{x} \in \R : 0 < x < 1}$ is infinite but does not have the same cardinality as $\N$ and also does not have the same cardinality as $\emph{E}$ or $\Z$. 
            \begin{proof}
                \item[\bf{Proof}:]
                    First I will start by defining (0,1): 
                    \begin{itemize}
                        \item The set (0,1) is all of the real numbers between 0 and 1 excluding the endpoints. There is no smallest or largest number cause for all real numbers in the set there will always be one in-between. (EX: 0.2 and 0.3 -> 0.25) This showcases that the set (0,1) is not infinite. even though you know the numbers are there, this set is still unaccountably infinite which I will show below:
                        \begin{itemize}
                            \item To show that the set (0,1) is unaccountably infinite I will use Cantor's Diagonalization Argument which proves that the set of numbers between 0,1 is unaccountably infinite while $\N$ is countably infinite. 
                            \item To prove Cantors's I am going to use contradiction with bijection. With bijection f : N $-> (0,1)$ would mean that we could list all the real numbers in (0,1) as f(s), f(2), f(3), etc...
                            \item I will have each number in (0,1) be represented by its decimal expansion where $a_{i,j}$ represents the j-th digit in the decimal expansion of the real number defined as f(i):
                            \begin{itemize}
                                \item $f(1) = 0.a_{1,1}, a_{1,2}, a_{1,3}, a_{1,4}, etc....$
                                \item $f(2) = 0.a_{2,1}, a_{2,2}, a_{2,3}, a_{2,4}, etc....$
                                \item $f(3) = 0.a_{3,1}, a_{3,2}, a_{3,3}, a_{3,4}, etc....$
                            \end{itemize}
                            \item Now I am going to make a new real number with a different decimal expansion than the previous. For this I am going to use n so its not random abritrary numbers. I was to ensure that the the new number differes from f(n) in the n-th digit for all n in $\N$. Which will show that r cant be on the list which contradicts f being a bijection and hence means that (0,1) does not have the same cardinality as $\N$. Shown below:
                            \begin{itemize}
                                \item $r_{\emph{n}} = \{{1 if a_{n,n} != 1 \atop 2 if a_{n,n,} = 1}$
                            \end{itemize}
                            \item Now I need to make sure that the cardinality of (0,1) is greater than $\N$. $\N$ is countably infinitewhile (0,1) is uncountably infinite. The cardinality of (0,1) is also greater than the cardinality of $\N$. since the cardinality of (0,1) is greater than the cardinality of $\N$ then $\N != (0,1)$. (Not the same)
                        \end{itemize}
                    \end{itemize}
            \end{proof}
        \item[\bf{Conclusion}:] The set (0,1) is infinite but doesn't have the same cardinality as $\N$, $\emph{E}$, and $\Z$ which are countably infinite. While (0,1) is uncountably infinite. \\
\noindent\rule{\textwidth}{0.4pt}
        \item Recall $\implies, \vee, \wedge,$ and $\nabla$ are binary logical operations on $\mathcal{M}$ we've studied. Recall $\neg$ is a (the only) logical \emph{operator} (or \emph{unary operation}) on $\mathcal{M}$ we've studied. Please determine, with justification, the number of different binary logical operations on $\mathcal{M}$ there can be. Also, determine the number of logical operators there can be.  
        \item[\bf{Claim}]: The set $\mathcal{M} = {0,1}$ has 16 possible binary logical operations and 4 possible unary logical operations. 
            \begin{proof}
                \begin{itemize}
                    \item The amount of different binary logical operations on $\mathcal{M}$ can be found with understanding the following:
                    \item A Binary Logical Operation takes two elements for the set $\mathcal{M}$ as inputs and produces a single output (also from $\mathcal{M}$). To find the number of operations we need to look at the function mappings for all the possible input pairs from $\mathcal{M}*\mathcal{M}$ to $\mathcal{M}$. First we will look at the input possibilities.
                    \begin{itemize}
                        \item If $\mathcal{M}$ has n elements, a binary operation requires a decision for each pair of inputs. Referring to positive and negative integers. so, the distinct number of input pairs can be defined using absolute values: $|\mathcal{M}| * |\mathcal{M}| = n^{2}$. I can do this because even if the input number is negative, a negative times a negative will still be a positive so this is a legal change. 
                        \item now I will show the Output possibilities for every pair of $\mathcal{M} * \mathcal{M}$ the operation has to produce an output in $\mathcal{M}$. And for each pair there are n possible outputs. 
                        \item So, since there are n possible outputs for each of the $n^{2}$ input pairs. Meaning that the total number of binary operations is $n^{n^{2}}$.
                        \item And Looking at the binary logical operations that we can take from the last question ( (0,1) ), $\mathcal{M} = 2^{2^{2}}$ which becomes, $2^{4}$. Then, 16. So, the number of logical binary operations on the set {0,1} is equal to 16. The operations include $\wedge$, $\vee$, $\implies$, $\longleftrightarrow$, and others. 
                    \end{itemize}
                    \item Now I will show how to get the unary logical operation on $\mathcal{M}$. a unary logical operation is defined as taking a single element from $\mathcal{M}$ as input and producing a single element from $\mathcal{M}$ as an output. so, basically one in one out. 
                        \begin{itemize}
                            \item First lets look at the input possibilities. If $\mathcal{M}$ has n elements, the operation needs to map each n input to an output in $\mathcal{M}$.
                            \item And now the output possibilities. For each input there is n possible outputs. the result has to be in $\mathcal{M}$.
                            \item Combining these two to find the total unary operators we get $n^{n}$. This is cause there are n outputs possible for each of n inputs. 
                            \item Similar to the logical operators will will use the set {0,1} on $\mathcal{M}$ to get $2^{2}$. Which will be equal to 4. I will show them below:
                            \begin{itemize}
                                \item The identity operation: id(x) = x
                                \item The negation operation: $\neg 0 = 1 and \neg 1 = 0$
                                \item A Constant operation that always returns 0: f(x) = 0 for all x in the set of $\mathcal{M}$
                                \item and a constant operation that always returns 1: f(x) = 1 for a x in the set $\mathcal{M}$.
                            \end{itemize}
                        \end{itemize}
                \end{itemize}
            \end{proof}
        \item[\bf{Conclusion}:] The number of binary logical operations on $\mathcal{M} = {0,1} is 16$ and the number of different unary logical operations on $\mathcal{M} = {0,1} is 4$\\
\noindent\rule{\textwidth}{0.4pt}
	\item Turn the logical system $(\mathcal{M}, \Phi, \wedge, \vee, \implies, \Longleftarrow, \Longleftrightarrow, \neg)$ into a purely algebraic system over $(\Z_2, +,\cdot)$ with, for $x \in \mathcal{M}$, $\Phi(x) = 0$ if $x$ is true and $\Phi(x) = 1$ if $x$ is false. Use the logical system to show that any statement of the form $[\neg P \wedge ((\neg P \implies Q) \wedge \neg Q)] \implies P$ is a tautology.
        \item[\bf{Claim}]: The logical statement $[\neg P \wedge ((\neg P \implies Q) \wedge \neg Q)] \implies P$ is a tautology (always true regardless of the truth values of $\P$ and $\Q$ when translated into an algebraic system over $\Z_{2}$
            \begin{proof}
                To begin proving this the first thing that I will do is map the logical operations to algebraic ones in $\Z_{2}$. 0 is true 1 is false. 
                \begin{itemize}
                    \item Beginning with Negation ($\neg$P). Negation P is the complement to $\Z_{2}$. For any $\P$ in the set of $\mathcal{M}$, and true of false/vise-versa. The algebraic definition $\neg$P = 1 + P (mod 2)
                    \item And Conjunction $(P \wedge Q)$ which is equivalent to multiplication. So, for any $P, Q, \in \mathcal{M}: P \wedge \Q = P * Q$.
                    \item Dis-junction $(P \vee Q)$ is equivalent to addition, (P+Q)
                    \item Implication $(P \implies Q)$ is equal to $\neg P \vee Q$. Which is $1 + P + P * Q$.
                    \item Biconditional $(P \Longleftrightarrow Q)$translates to $1 + P + Q$ (mod 2)
                \end{itemize}
                Now that the logical system is in algebraic terms we are going to prove the tautology. I am gonna translate and simplify the statement into the logical equivalent.
                \begin{itemize}
                    \item $\neg$P becomes 1 +P
                    \item $\neg P \implies Q$ becomes $1 + (1 + P) + (1 + P) * Q$ which equals $P + P * Q$
                    \item $\neg$Q becomes 1 + Q
                    \item P stays as P
                \end{itemize}
                So, $[\neg P \wedge ((\neg P \implies Q) \wedge \neg Q)] \implies P$ becomes $[(1 + P) * ( P + P * Q) * (1 + Q)] \implies P$ \\
                Now I expand this...
                \begin{itemize}
                    \item $(1 + P) * ( P + P * Q)$ becomes $P + P * Q + P * (P + P * Q)$ then $ P + P * Q$
                    \item (Note: P * P = P and P * P * Q = P * Q)
                    \item And Multiply by 1 + Q... $P + P * Q) * (1 * Q) = P * (1 + Q) + P * Q * (1 + Q) = P + P * Q + P * Q * (1 + Q)$. Since P * Q * (1 + Q) = 0, (cause Q * (1 + Q) = 0 in $\Z_{2}$) it becomes P + P * Q.
                    \item so next we do the implication for (P + P * Q) implies P.
                    \begin{itemize}
                        \item First we must note that the algebraic form of implication A implies B is 1 + A + A * B. 
                        \item Applying it we get (P + P * Q) implies P = 1 + (P + P + Q) + (P + P * Q) which expands to become: 1 + P + P * Q + (P + P * Q) * P = 1 + P + P * Q + P + P * Q which equals 1. (cause P + P = 0 and P * Q + P * Q = 0)
                    \end{itemize}
                \end{itemize}
            \end{proof} \\
        \item[\bf{Conclusion}:] So, the final expression being equal to 1 proves that the original value is always true no matter what the arbitrary values of P and Q are.  \\
\noindent\rule{\textwidth}{0.4pt}
    \end{enumerate}	
\end{document} 
